% Created 2018-02-21 Wed 19:30
% Intended LaTeX compiler: pdflatex
% \documentclass[11pt]{cu_thesis}
\documentclass[journal,12pt,onecolumn,draftclsnofoot,,twoside]{IEEEtran/IEEEtran}
\usepackage[utf8]{inputenc}
\usepackage[T1]{fontenc}
\usepackage{graphicx}
\usepackage{grffile}
\usepackage{longtable}
\usepackage{wrapfig}
\usepackage{rotating}
% \usepackage[normalem]{ulem}
\usepackage{amsmath}
\usepackage{textcomp}
\usepackage{amssymb}
\usepackage{capt-of}
\usepackage{hyperref}
\usepackage[inkscapelatex=false, inkscapepath=svgsubpath]{svg}
\usepackage{subfigure}
\usepackage{xspace, dsfont}
\usepackage{mathtools}
\newcommand{\Dm}{\ensuremath{\mathcal{D}_{-} }\xspace}
\newcommand{\U}{\ensuremath{\mathcal{U} }\xspace}
\newcommand{\DU}{\ensuremath{\Delta \mathcal{U} }\xspace}
\newcommand{\Q}{\ensuremath{\mathcal{Q} }\xspace}
\newcommand{\R}{\ensuremath{\mathcal{R} }\xspace}
\newcommand{\Hh}{\ensuremath{\mathcal{H} }\xspace}
\newcommand{\du}{\ensuremath{\Delta u }\xspace}
\newcommand{\dU}{\ensuremath{\Delta \mathcal{u} }\xspace}
\newcommand{\Gd}{\ensuremath{\tilde G }\xspace}
\newcommand{\Ad}{\ensuremath{\tilde A }\xspace}
\newcommand{\Bd}{\ensuremath{\tilde B }\xspace}
\newcommand{\Cd}{\ensuremath{\tilde C }\xspace}
\newcommand{\xd}{\ensuremath{\tilde \xi }\xspace}
\newcommand{\Qd}{\ensuremath{\tilde Q }\xspace}
\newcommand{\Rd}{\ensuremath{\tilde R }\xspace}
\newcommand{\x}{\ensuremath{\xi }\xspace}
\newcommand{\xdss}{\ensuremath{\tilde \xi_{ss} }\xspace}
\newcommand{\xde}{\ensuremath{\tilde \xi_{e} }\xspace}
\newcommand{\xo}{\ensuremath{\hat \xi }\xspace}
\newcommand{\yr}{\ensuremath{\nu_{r} }\xspace}
\newcommand{\y}{\ensuremath{\nu} \xspace}
\newcommand{\dd}{\ensuremath{\Delta }\xspace}
\newcommand{\ub}{\ensuremath{\bar{u} }\xspace}
\newcommand{\ubp}{\ensuremath{\bar{u}^+ }\xspace}
\newcommand{\ubm}{\ensuremath{\bar{u}^- }\xspace}
\author{arnold}
\date{\today}
\title{}
\hypersetup{
 pdfauthor={arnold},
 pdftitle={},
 pdfkeywords={},
 pdfsubject={},
 pdfcreator={Emacs 24.4.1 (Org mode 9.1.4)}, 
 pdflang={English}}
\begin{document}



\section{Introduction}
\label{sec:orge362f71}

In the context of CS, we are interested in rapidly moving the AFM stage between setpoints. The setpoints define either discrete measurement locations or the start of a new \(\mu\) -path scan. In earlier work, we found while exploring the limits of achievable with plain linear feedback that the change in the control input, \(\Delta u(k)\) was the limiting factor. We  \cite{braker_application_2017}, \cite{braker_fast_2017} took the viewpoint of trying to achieve a stable system for a equivalent given feedback gain in the face of rate-of change saturation. The methods we chose for doing this were using model predictive control (MPC) and tracking an optimal trajectory generated by solving a contrained, linear quadratic regulator problem (CLQR). In both cases, we considered a standard quadratic cost function

\begin{align}
V(z,v) &= \min_{v} z_{N}Q_{p}z_{N} + \sum_{i=0}^{N-1}z_{i}^{T}Qz_{i} + v^{T}_{i}Rv_{i}\\
 \text{s.t.}&\\
z_{i+1} &= Az_{i} + Bv_{i}\\
z_{0} &= x_{k}\\
|v_i - v_{i-1}| &\coloneqq |\Delta v(i)| \leq (\Delta u)_{max}
\label{eqn:opt}
\end{align}


The essential difference between the MPC scheme and the trajectory tracking scheme is that the optimzition in MPC solved repetaly online, and the control input given to the system is \(v(0)\). In the trajectory tracking scheme, the optimization is solved offline and the resulting optimal input trajectory is fed to the system in a feedforward manner while state feedback tracks the optimal state sequence. In both cases, the weight \(Q, R)\) were chosen via an inverse optimal control problem, such that the associated linear feedback (in the infinite horizon, unconstrained LQR problem) would place the poles with the desired damping. 

It was our conclusion that both methods are effective at extending the stable range of a given \((Q,R)\) pair in the face of \(\Delta u\) saturation. The attractiveness of MPC here is that it does not depend on defining a list a setpoints to visit a-priori. The downside to MPC is that to significantly expand the region of rechable setpoints requires a large control horizon. Our MPC implementation used the Fast Graident Method (FGM) to solve the quadratic program associated with \eqref{eqn:opt}. Unfortunately, the convergence rate of the FGM depends heavily on the condition number of the Hessian matrix of the QP. Several factors contribute to a poorly conditioned Hessian matrx: (1) long control horizon (2), aggressive (Q,R) weights. The other factor limiting the horizon length in MPC is that the problem size grows, making the QP more difficult to solve within the required sampletime. 

The question which naturually arises from our prior work and which is the main topic of this study, is "how much \emph{time}" is actually saved with either MPC or CLQR tracking?" In essence, the reachable set of setpoints using pure linear feedback can be expanded by de-rating the linear feedback gain. While we might expect this to result in a slower settling time, the question we seek to answer is \emph{how much slower}? In short, are the more complicated methods worth it?

A second related question is \emph{if you're going to go through the trouble of generating and tracking an optimal trajectory, why not track the time-optimal trajectory?}

We will answer both of these questions in this paper. 




\section{Preliminarys}
\label{sec:orgbc45dd3}
\subsection{Modeling}
\begin{figure}
  \includegraphics[scale=0.7]{figures/exp_setup.pdf}
  \caption{Experimental setup}
  \label{fig:exp_setup}
\end{figure}
The AFM in our lab consists of an Agilent 5400 which has been retrofited with an nPoint NPXY100A piezo stage, which provides lateral movement of the sample. which is driven by a C300 signal conditioner/amplifier. The C300 amplifies the low voltage ($\pm~10$ volts) control inputs to a voltage signal which drives the piezo stage and provides signal conditioning for the capacitive position sensors in the stage. All control logic is programmed into a Xilinx FPGA in a cRIO 9082 from National Instruments.


To keep the discussion manageable, we will concentrate the discussion to the $x$-direction.  There are enough things to compare without adding another axis. A frequency response function (FRF) is shown as the red curve in Fig.~\ref{fig:frf_xdir}. The experimental FRF is obtained using a stepped sines (single frequency at a time) method. 

The black dashed curve in Fig.~\ref{fig:frf_xdir} is a state space model with 22 states, 9 of which model transport delay.

\begin{figure}[ht!]
  \centering
  \includesvg[scale=1]{figures/frf_xdir.svg}
  \caption{Frequency response of the piezo stage in the $x$-direction. The dashed black curve is a 12th order model fit to the data using a subspace realization method.}
    \label{fig:frf_xdir}
\end{figure}


\subsection{Limitation on control voltage rate of change}
\begin{figure}
  \centering
  \includesvg[scale=1]{figures/G_uz2pow.svg}
  \caption{Frequency response from control input to high voltage signal (blue) and from high voltage input to stage position (red)}
  \label{fig:powfrf}
\end{figure}

The high voltage output of the C300 is current limited to 100 mA. Because the piezo stage presents as a capacitive load to the power amplifier, this translates to a rate of change limitation on the voltage control signal. To see this, we have $\dot v_{h} = \frac{1}{C} i_c(t)$, or, with a backwards difference approximation
\begin{equation}
  |\frac{V_{h}(k+1) - V_{h}(k)}{T_s}| \leq \frac{1}{C} 100~\text{mA}
\end{equation}
Note that this is the rate of change limit on the high voltage signal, not the low voltage control input signal. Naively, the low voltage rate of change limitation is
\begin{align}
K_{pow}|\frac{V_{\ell}(k+1) - V_{\ell}(k)}{T_s}|  &= |\frac{V_{h}(k+1) - V_{h}(k)}{T_s}| \leq \frac{1}{C} 100~\text{mA}\\
|V_{\ell}(k+1) - V_{\ell}(k)|  &\leq \frac{Ts}{K_{pow}C} 100~\text{mA}
\end{align}
where $K_{pow} = 180/20 = 9$ and $v_{\ell}$ is the low voltage control input. However, as usual things are more complicated. At some point along the path from low voltage input to high voltage output, the C300 low pass filters the signal. We can identify this LPF by measuring the frequency response from low voltage input to high voltage output. This is shown in Fig.~\ref{fig:powfrf}

It is useful and thus interesting that in discrete time, the rate of change dynamics for linear system are the same as the standard input dynamics. Consider a state space system $G=\{A,B,C\}$ with state $x(k)$ and output $y(k)$. Then
\begin{align}
  \Delta y(k) &\coloneqq y(k) - y(k-1) = C\Delta x(k) = C(x(k) - x(k-1))\\
  \Delta x(k+1) &= x(k+1) - x(k) = Ax(k) + B u(k) - [Ax(k-1) + Bu(k-1)]\\
              &= A \Delta x(k) + B\Delta u(k).
\end{align}



\subsection{The LQR Optimal Control problems}
In the optimizations we consider, it will often be useful to work with an incremental form of \(G\) which has as its input \({\Delta u(k)\coloneqq u(k)-u(k-1)}\), rather than \(u\), which is defined by augmenting \(G_p\) with an state \x\(_{\text{u}}\)(k) such that
\begin{equation*}
  \x_u(k) = u(k-1).
\end{equation*}
It follows that
\begin{subequations}
\begin{align}
  \xd(k+1)
  &=
    \begin{bmatrix}
      A & B\\ 0 & I
    \end{bmatrix}
    \xd(k)
    +
    \begin{bmatrix}
      B\\I
    \end{bmatrix}
  \Delta u(k) \\
  \y(k) & = \begin{bmatrix}C & 0\end{bmatrix}\xd(k)\\
  \xd(k)& \coloneqq \begin{bmatrix}\x(k)\\\x_u(k) \end{bmatrix}\\
  \xd(0) & = \begin{bmatrix}\x(0)\\u(-1)\end{bmatrix}. \label{eqn:x0_aug}
\end{align}\label{eqn:ssdelta}%
\end{subequations}
We call this system \(\Gd = \{\Ad, \Bd, \Cd, 0\}\), which has \({\tilde{n}_s=50}\) states.
To solve the setpoint tracking problem, we work in the error
coordinates of \(\Gd\).
For an arbitrary reference \(\y_r\), in steady state we have \({\du_{ss}=0}\) and \(\xdss =N_{\xi}\y_r\) where \({N_{\xi}\in\mathds{R}^{\tilde{n}_s\times n_u} }\) is found by solving
\begin{align}
  \begin{bmatrix}N_{\xi} \\ N_u\end{bmatrix} &=
\begin{bmatrix}I-\Ad & -\Bd\\\Cd & 0\end{bmatrix}^{-1}\begin{bmatrix}0\\ I\\\end{bmatrix}\label{eqn:nxnu},
\end{align}
which will give \(N_u\equiv 0\).
The error state, \({\xde(k)=\xd(k) - \xdss}\) has dynamics
\begin{align}
  \xde(k+1) & = \Ad\xd(k) + \Bd\dd u(k) - \xdss \nonumber\\
            & = \Ad \xde(k)   + \Bd \dd u(k)\nonumber.
\end{align}



\subsection{The minimum-time control problem in discrete-time}
\label{sec:org6d1d6b6}
In discrete-time, the minimum-time control problem can be stated as \cite{chen_minimumtime_cca}
\begin{align}
\min_{u(0), u(1),\dots,u(N-1)} & N\\
\text{s.t.}&\\
x_{k+1} & = Ax_{k} + Bu_{k}\\
x_{T} & = x_{f}\\
u_{k}&\in \mathds{K},~k=0,\dots,N-1\\
x_{k} &\neq x_{f},~k=0,\dots,N-1.
\end{align}

More practically, the problem can be solved using a bisection method which searches for the smallest feasible \(N\). In these cases, the bisection search solves sub-problem given by


\begin{align}
\min_{U}& || x_{f} - x_{N}||\\
\text{s.t.} &\\
x_{k+1} & = Ax_{k} + Bu_{k}\\
u_{k}&\in \mathds{K},~k=0,\dots,N-1\\
\end{align}

For a given \(N\), if \(||x_{f} - x_{N}|| < \text{TOL}\), the sub-problem returns successful, otherwise it fails. The goal then is to find a the smallest \(N\) that succeeds. 

It is also worth pointing out that the solution to this problem, in contrast to continuous time systems is not in general bang-bang CITE.

When we consider comparing the settling time of our LQR inspired methods to the time-optimal solution, we need to be careful.
The reason is that the time optimal solution requires all the states to have reached steady at $x_f$. Consequently, the total move, if we compute the time-optimal trajectory for $G$, will be dominated by the slow pole at 220 Hz. This is illustrated in Fig.~\ref{fig:slowpz_to}.

\begin{figure}
  \centering
  \includesvg[scale=1]{figures/timeopt_slowpz_illustrate_y.svg}
  \includesvg[scale=1]{figures/timeopt_slowpz_illustrate_u.svg}
  \caption{The problem with computing the time-optimal trajectory when the model has the slow PZ-pair.}
  \label{fig:slowpz_to}
\end{figure}
In some sense, the slow pole-zero pair is a first order drift model, so we call it $G_{drift}$. Then, the total transfer function from the control to the piezo stage position can be factored as $G(z) = G_{drift}\hat{G}(z)$. The idea then, is to compute the time-optimal trajectory for $\hat{G}(z)$. This results in optimal trajectories for the control input, $u^*$ and plant output $y^*$ for $\hat{G}$. That optimal control sequence is will then be run through a feedforward filter which inverts the dynamics of $G_{drift}$. The scheme is illustrated in Fig.~\ref{fig:bd_ff_to}
\begin{figure}
  \centering
  \includegraphics[scale=0.5]{figures/bd_ff_timeopt.pdf}
  \caption{Time optimal trajectory with selected plant inversion.}
  \label{fig:bd_ff_to}
\end{figure}


\section{Quantifying the time savings (if any) of MPC and d-rated linear feedback}
\label{sec:org7edf36d}
The goal of this section is to quantify the time savings (if any) of using MPC vs SLF. While is is certainly true that for a given feedback gain (i.e., $(Q,R)$ pair), we can extend the stable range of setpoints. However, that does not tell us how much time we save with that strategy. To be able to make this comparison, we need to be able to decide when a trajectory is unstable, or more generally, unacceptable. The most niave way to this would be to just look at the settling time over a sufficiently long simulation: if the trajectory never settles into the settle boundary, we decide the system was unstable for that setpoint size. The trouble with this is that there is a middle ground, where the trajectory will actually settle, but spends a lot of time oscillating (in some non-linear fashion) before it gets there. This is illustrated in Fig. REF.


\subsection{Rejection Metrics}
There are several ways we might think about doing this. 
For a given $\gamma$, we need a way to decide if the trajectory which results from either SLF or MPC is acceptable for a given setpoint size.

Ultimately, we have decided to do is to compare the trajectory resulting from MPC or SLF with the one from CLQR: if the settling time is greater than 20\% of the CLQR trajectory, we decide that the current setpoint was too large.


\begin{enumerate}
\item We start with a nominal \((Q,R)\) pair.
\item We determine how large we must make \(\gamma R\) to visit a certain size of setpoint. In general, the larger \(\gamma\) is, the larger the maximum setpoint we can visit is. This is what I mean by "de-rating" the feedback gain and is illustrated below in Figur.
\end{enumerate}


Note that each point in that plot was generated by simulated the linear feedback law with saturation starting from a very small setpoint and incrementing larger (by 0.1 I think) until the output becomes unstable. 

So the questions I want to answer are:
\begin{enumerate}
\item Given a maximum setpoint \(ref_{max}\), how much time do we \textbf{loose} by making \(\gamma\) large? Said another way, what is the time penalalty for needing to visit larger setpoints (since this implies that \(K\) must be "smaller")?
\item How does this time compare to what we can acheive with MPC?
\end{enumerate}


The way I think it makes sense to think about these questions is to compare the settling times to the optimal, constrained finite horizon (CLQR) open-loop settling time. At least for a given quadratic cost, that is typically the best settling time we can hope to achieve, and we won't run into the problems with MPC where need a larger control horizon to get stability. 

\begin{figure}
  \centering
  \subfigure[The increase in maximum achievable setpoint as the control weight $\gamma$ increases for both saturated linear feedback different MPC control horizons.
  \label{fig:maxsp_by_gam}]
  {\includesvg[width=3in]{figures/maxref_vs_gamma_dumaxp6.svg}}
\hfill
\subfigure[Comparison of settling time the minimum-time solution and the CLQR solution.
\label{fig:ts_clqr_to}]
{\includesvg[width=3in]{figures/clqrTimeOpt_sp_vs_ts_CCTA_dumaxp6.svg}}
% \caption{both}
\end{figure}

The result of running a parameter sweep this way is shown in Fig.~\ref{fig:maxsp_by_gam}. What the figure shows is, for CLF, and several values of the MPC control horizon, the maximum setpoint that can be tracked for a given control weight $\gamma$. In the figure $\gamma$ ranges from 10 to 5000 in XX increments. For each value of $\gamma$, we performed successively larger step-input simulations between and 12 v. The $y$-axis value indicates that that value of reference resulted in a deteriorated or unstable trajectory, and is thus the largest track-able setpoint for the given gamma.

\section{Time Savings Analysis}
The goal of doing this is to essentially look at several cross sections Fig.~\ref{fig:maxsp_by_gam} and look at how the settling time changes between CLF and different MPC control horizons \emph{for a given desired maximum setpoint.}

Here, we choose several desired $r_{max}$ values and consider how much time is saved over the time optimal solution for the different strategies. We choose $r_{max}$ as 2.5, 5.0 and 10. These values are illustrated in Fig.~\ref{fig:maxsp_by_gam} by the dotted black lines. Where those dotted black lines cross the various curves indicates the $\gamma$ that is required.

\begin{figure*}
  \begin{minipage}{0.32\textwidth}
    \includesvg[width=1.2\textwidth]{figures/perc_increase_rmax2p5.svg}
  \end{minipage}
    \begin{minipage}{0.32\textwidth}
    \includesvg[width=1.2\textwidth]{figures/perc_increase_rmax5.svg}
  \end{minipage}
  \begin{minipage}{0.32\textwidth}
    \includesvg[width=1.2\textwidth]{figures/perc_increase_rmax10.svg}
  \end{minipage}
  \caption{The percentage increase over the time-optimal solution for CLF (orange), the CLQR solution (blue) and the MPC strategy for control horizons ranging from $N=4$ to $N=20$. }
  \label{fig:perc_inc}
\end{figure*}
Our takeaway from this study is that there appears to very little benefit in terms of time saved for implemented the CLQR or MPC control strategies. The largest difference in the percent increase is only in the range of 10\%, and for the smaller $r_{max}=2.5$, this difference is even smaller. 

\section{Experimental Comparison}
Our simulation results indicate that we may as well use CLF or track the time-optimal trajectory, if we can get away with it. But what about the real world?

\newpage

% \FloatBarrier
\section{Imaging Results}
In these section, we demonstrate using the time-optimal tracking method to actually acquire images via $\mu$-path based CS.
\begin{figure}
  \centering
\includegraphics[scale=1]{figures/5micron_rasterscans_v2.eps}
\includegraphics[scale=1]{figures/5micron_csscans_v2.pdf}
\caption{RESULT!}
\label{fig:imaging}
\end{figure}

\section{Bibliography}
\label{sec:org5b4a114}

\bibliographystyle{IEEEtran}
\bibliography{/home/arnold/bib_pdf/main_bibliography.bib}
\end{document}