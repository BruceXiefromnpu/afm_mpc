% Created 2018-02-21 Wed 19:30
% Intended LaTeX compiler: pdflatex
\documentclass[journal,twocolumn,twoside]{IEEEtran}
% \documentclass[journal,12pt,onecolumn,draftclsnofoot,,twoside]{IEEEtran/IEEEtran}
\pdfminorversion 4 % This is so manuscript central can use it.

\usepackage[utf8]{inputenc}
% \usepackage[T1]{fontenc}
\usepackage{graphicx}
\usepackage{grffile}
\usepackage{longtable}
\usepackage{wrapfig}
\usepackage{rotating}
\usepackage{booktabs}
% \usepackage[normalem]{ulem}
\usepackage{amsmath}
\usepackage{textcomp}
\usepackage{amssymb}
\usepackage{capt-of}
% \usepackage{hyperref}
\usepackage[inkscapelatex=false, inkscapepath=svgsubpath]{svg}
\usepackage{subfigure}
\usepackage{xspace, dsfont}
\usepackage{mathtools}
\usepackage{multirow}

\newcommand{\Dm}{\ensuremath{\mathcal{D}_{-} }\xspace}
\newcommand{\U}{\ensuremath{\mathcal{U} }\xspace}
\newcommand{\DU}{\ensuremath{\Delta \mathcal{U} }\xspace}
\newcommand{\Q}{\ensuremath{\mathcal{Q} }\xspace}
\newcommand{\R}{\ensuremath{\mathcal{R} }\xspace}
\newcommand{\Hh}{\ensuremath{\mathcal{H} }\xspace}
\newcommand{\du}{\ensuremath{\Delta u }\xspace}
\newcommand{\dU}{\ensuremath{\Delta \mathcal{u} }\xspace}
\newcommand{\Gd}{\ensuremath{\tilde G }\xspace}
\newcommand{\Ad}{\ensuremath{\tilde A }\xspace}
\newcommand{\Bd}{\ensuremath{\tilde B }\xspace}
\newcommand{\Cd}{\ensuremath{\tilde C }\xspace}
\newcommand{\xd}{\ensuremath{\tilde x }\xspace}
\newcommand{\Qd}{\ensuremath{\tilde Q }\xspace}
\newcommand{\Rd}{\ensuremath{\tilde R }\xspace}
\newcommand{\x}{\ensuremath{x }\xspace}
\newcommand{\xdss}{\ensuremath{\tilde x_{ss} }\xspace}
\newcommand{\xde}{\ensuremath{\tilde x_{e} }\xspace}
\newcommand{\xo}{\ensuremath{\hat x }\xspace}
\newcommand{\yr}{\ensuremath{y_{r} }\xspace}
\newcommand{\y}{\ensuremath{y} \xspace}
\newcommand{\dd}{\ensuremath{\Delta }\xspace}
\newcommand{\ub}{\ensuremath{\bar{u} }\xspace}
\newcommand{\ubp}{\ensuremath{\bar{u}^+ }\xspace}
\newcommand{\ubm}{\ensuremath{\bar{u}^- }\xspace}

\begin{document}
\title{Control strategies for step tracking a piezo stage}
\author{Roger A. Braker and Lucy Y. Pao
  \thanks{The authors are with the Dept. of Electrical, Computer, and Energy Engineering at the University of Colorado, 425 UCB, Boulder, CO 80309, United States. Phone: +1 (303) 492-2360. Fax: +1 (303) 492-2758.
    R. A.  Braker (corresponding author roger.braker@colorado.edu) is a graduate student and
    L.Y. Pao (pao@colorado.edu) is the Richard \& Joy Dorf Professor.}
  \thanks{This work was supported in part by the US National Science Foundation (NSF Grant CMMI-1234980), Agilent Technologies, Inc., and the Hanse Wissenschaftskolleg in Delmenhorst, Germany.}
}

\maketitle
\begin{abstract}
  This paper considers the setpoint tracking performance of piezo nano-positioning stage subject to rate of change limitations of the control signal imposed by the power amplifier. We compare the settling-time performance of a Model Predictive Control scheme to saturated linear feedback to compensate the vibrational dynamics of the stage. In both cases, hysteresis and drift are compensated via dynamic inversion. We show how the 
\end{abstract}


\section{Introduction}
\label{sec:orge362f71}
BOILER PLATE ABOUT AFM AND COMPRESSED SENSING.
 \newline

In the context of CS, we are interested in rapidly moving the AFM stage between setpoints. The setpoints define either discrete measurement locations or the start of a new \(\mu\)-path scan.

Point-to-point movements by AFM is also of interest in other areas like visceolastic property mapping \cite{killgore_visceolastic_2011}.\newline

BOILER PLATE ABOUT MINIMUM TIME STUFF

Reference \cite{broeck_time_2009} uses a quadratic cost with a terminal constraint combined with an exhaustive search over the control horizon length to formulate a Time Optimal MPC problem. However, the implementation only runs at 100~Hz, which is about 250 times too slow for our application. 
\newline

One of the primary constraints in setpoint is tracking is the power amplifiers current which roughly translates to a slew rate limitation on the control signal. In earlier work, we considered the use of model predictive control to directly address the rate of change limitation and showed that the range of stabilizable setpoints was increased as compared to a simple saturated linear feedback \cite{braker_application_2017}. The question which naturally arises from that study, and which is one of the main topics of this paper, is "how much \emph{time} is actually saved with MPC?" In essence, the stabilizable set of setpoints using saturated linear feedback can be expanded by relaxing the performance criteria. While we might expect this to result in a slower settling time, the question we seek to answer is \emph{how much slower}? In other words, given a range of setpoints we wish to be able to visit, to what extent does MPC reduce the settling time and is this reduction significant enough to warrant the additional complexity?

The second limitation of \cite{braker_application_2017} is that we did not consider the affects of drift and hysteresis and only considered tracking a single setpoint with the stage starting at rest. When tracking a sequence of setpoints, the affects of hysteresis become much more prominent. Thus, in this paper, we employ inverse drift and hysteresis compensation. Those inversions have the potential to violate the slew-rate limit our MPC controller has worked so hard to respect. To counteract this, in Section REF, we derive bounds on the rate of change of the output of the inverse compensators such that the slew rate limit can be guaranteed.

As we show in Section \ref{sec:powcharct}, this limitation is a function of the transfer function from the low voltage control signal $u(k)$ to the current output of the power amplifier. To simplify the constraints, we derive a simple bound using frequency domain arguments.

The overall control structure we consider in this paper is illustrated in block diagram in Fig. ~\ref{fig:ss_bd}. We consider the plant to be a cascaded model of hysteresis ($\mathcal{H}$),  drift ($G_d$) and vibrational dynamics ($G_{vib}$). Modeling these three systems are the subject of Sections \ref{sec:hyst_model}, \ref{sec:drift_model}, and \ref{sec:vib_model} respectively. Generally speaking, all higher frequency dynamics, including transport delay, are collected into the vibrational model.




We want to look at a few different ways of driving the stage from point to point as rapidly as possible, yet still be practical. Inspired by our past work, we focus on the experimental comparison of two methods:
\begin{itemize}
\item\emph{Saturated Linear state feedback (SLF)} The basic idea here to use standard linear state feedback which is saturated (on $\Delta u$) but to de-rate the design such that the closed-loop system is demonstrably stable up to a certain reference size. 
\item\emph{Constrained Model Predictive Control (MPC)} The goal with MPC is that, instead of fully de-rating the design to account for the slew rate constraint, to directly account for the constraint as part of the control law itself. Do to limitations on the implementation, MPC must still be de-rated to some extent to be implementable, since the length of control horizon needed to achieve stabilty over the full range of the stage becomes large for an aggressive design.
\end{itemize}


\begin{figure*}
  \centering  
  \includegraphics[width=1\textwidth]{figures/blocks/ss_block_diagram.pdf}
  % \includegraphics[width=1\linewidth]{figures/Hyst_gdrift_gvib.pdf}
  \caption{The overall plant model consists of a vibrational component, $G_{vib}$, a drift model $G_{d}$ and a hysteresis model $\mathcal{H}[\cdot]$. }
  \label{fig:ss_bd}
\end{figure*}

\section{Experimental Testbed}\label{sec:testbed}
% \begin{figure*}
%   \begin{minipage}{0.48\textwidth}
%     \includegraphics[width=1\textwidth]{figures/exp_setup.pdf}
%     \caption{Experimental setup}
%     \label{fig:exp_setup}
%   \end{minipage}
%   \hfill
%   \begin{minipage}{0.48\textwidth}
%     \includegraphics[width=1\textwidth]{figures/c300_measurement_schematic.pdf}
%     \caption{Schematic of the augmented high voltage output measurement and current measurement used in characterizing the C300.}
%     \label{fig:c300_meas}
%   \end{minipage}
  
% \end{figure*}

The AFM in our lab consists of an Agilent 5400 which has been retrofited with an nPoint NPXY100A piezo stage, which provides lateral movement of the sample. The NPXY100A is driven by an nPoint C300. The C300 amplifies the low voltage ($\pm~10$ volts) control inputs to a high voltage signal which drives the piezo stage and provides signal conditioning for the capacitive position sensors in the stage. All control logic is programmed into a Xilinx FPGA in a cRIO 9082 from National Instruments.

In characterizing the limitations of this system, it will be helpful to enable direct measurements of both the high voltage output, $V_X$, and current, $I_X$, of the C300. These are obtained by re-routing the C300 drive signal as shown in Fig.~\ref{fig:c300_meas}. The voltage divider output provides a (scaled) measurement of the C300 output voltage and is given by
\begin{equation}
V_X = \frac{R_1 + R_2}{R_2}V_{div}
\end{equation}
Note that the impedance of the divider is high (about $32.5\text{M}\Omega$) to avoid perturbing the amplifier output.

A small, $0.1~\Omega$ resistor on the return side of the piezo provides a current measurement of $I_X$. The voltage across this resistor is amplified by an op-amp circuit so that
\begin{equation}
I_{X} = \frac{V_{op}}{K_{op}R_{\text{sense}}}
\end{equation}
where $K_{op}$ is the gain of the operational amplifier.

\section{System Modeling}
To keep the discussion manageable, we will concentrate the discussion to the $x$-direction. We will model the overall plant for the $X$-direction as three cascaded systems: $\mathcal{H}$ which models the hysteresis of the piezo, a drift  $G_d$ and a vibrational model $G_{vib}$. In general, the effects of hysteresis are most noticeable when moving across wide ranges. Thus, by using relatively small input signals, the drift and vibrational dynamics can be identified separately from the hysteresis. Here, we model both the $G_{vib}$ and $G_d$ as linear time-invariant systems. The modes of the vibrational dynamics are over an order of magnitude faster than the modes in the linear drift mode, which allows the two systems to be easily separted in the identification.  the drift as a linear systemThe dynamics of drift are predominantly low frequency while vibrational aspects on the other hand are fast by comparison. These differences can be exploited to develop each model separately. This cascaded structure is shown in \ref{fig:whole_plant}. Modeling these three components is the subject of the next three subsections.




% \begin{figure*}
%   \centering
%   \begin{minipage}{0.45\textwidth}
%   \includesvg[scale=01]{figures/G_uz2pow.svg}
%   \caption{Frequency response from control input to high voltage signal (blue) and from high voltage input to stage position (red). The red curve is the FRF for both the vibrational and drift dynamics.}
%   \label{fig:powfrf}
% \end{minipage}
% \hfill
% \begin{minipage}{0.45\textwidth}
%   \centering
%   \includesvg[scale=0.8]{figures/frf_xdir.svg}
%   \caption{Frequency response of the piezo stage in the $x$-direction. The dashed black curve is a 12th order model fit to the data using a subspace realization method.}
%   \label{fig:frf_xdir}
% \end{minipage}
% \end{figure*}

\subsection{Modeling $G_{vib}$}\label{sec:vib_model}
Here, we are concerned with the transfer function from the low voltage control signal to the stage's position, $G_{X,u_X}$. To obtain an experimental frequency response of $G_{X,u_X}$, we use a stepped-sines method (single frequency at a time). The amplitude of the driving sinusoid is chosen to be small enough that the effects of hysteresis are minimized. After the system reaches steady-state, the input and output signals are demodulated into their first (complex) Fourier coefficients, the ratio of which yields the frequency response at that frequency. One advantage of the stepped-sines methods is that we can consider the normalized residual energy in the output signal
\begin{equation}
E_r(\omega) = \frac{\sum_{k} |y_k|^2 - |Y_1(\omega)|^2}{\sum_{k} |y_k|^2}
\end{equation}
where $Y_1(\omega)$ is the first Fourier coefficient and $y_k$ is the output signal at time $k$. A very small $E_r$ indicates that nearly all of the energy is contained in $Y_1(\omega)$ and therefore that the system response is largely described by linear dynamics. Conversely, a large $E_r$ indicates that $y_k$ is corrupted by either noise or non-linearities. 

\begin{figure}
  \centering
  \includesvg[width=01\linewidth]{figures/G_uz2stage_Eres.svg}
  \caption{The solid red curve (left axis) is the frequency response from control input to stage position output in the $X$ direction. The dotted-black curve is the normalized residual energy in the output signal $y_X$.}
  \label{fig:Guz2stage_frf}
\end{figure}
% However, to characterize the limitations of the C300 it will also be helpful to consider the transfer functions enumerated in Table~\ref{tab:TFS}. 
% Note the large amounts of phase lag in both $G_{pow}$ and $G_{stage}$. This is due to the C300 including its own DSP which cannot be turned off, even when run in open loop mode. 

% \begin{table}
%   \centering
%   \begin{tabular}{ll}
%     name & description\\
%     \hline
%     \(G_{y_{X},u_{X}}\) & \multirow{2}{2.5in}{T.F. from low voltage control to stage position.}\\
%     &\\
%     \(G_{y_{X},u_{X}}\) & \multirow{2}{2.5in}{T.F. from high-voltage C300 output voltage control to stage position.}\\
%     &\\
%     \(G_{I_{X},u_{X}}\) & \multirow{2}{2.5in}{T.F. from low voltage control to C300 output current.}\\
%     &\\
%     \(G_{I_{X},V_{X}}\) & \multirow{2}{2.5in}{T.F. from high-voltage C300 output to C300 output current.}\\
%   \end{tabular}
%   \label{tab:TFS}
% \end{table}


To obtain a parametric models of $G_{X,u_X}$ for control design involves two steps. We obtain a preliminary model using an Eigenspace Realization Algorithm (ERA). In general the ERA will not produce a model with poles at $z=0$, which is what we need in order to model the delay. Thus, the delay in the frequency response is divided out before passing it to the ERA algorithm. The ERA does a fair job of identifying the resonances and anti-resonances. However, it gets the relative degree wrong.

Thus, the second step is to use the model generated by the ERA as the initial guess to a non-linear least squares problem \cite{sidman_parametric_1991} which minimizes the ratio of the logarithm of the experimental frequency response to that of the model. Though Sidman et al. develop the idea for a continuous time models, their strategy is easily adapted to fit a discrete time model. The optimization is given by
\begin{align}
\min_{\theta} \sum_{i=1}^M| \log(G(e^{j\omega_iT_s})) - \log(\hat{G}(e^{j\omega_iT_s}|\theta))|^2
\label{eqn:logfit}
\end{align}
where $\omega_i$ is each frequency in the experimental frequency response and $\hat{G}(e^{j\omega_iT_s}|\theta)$ is the model parametrized by the vector $\theta$. $\hat{G}(z|\theta)$ is composed of first and second-order factors
\begin{equation}
  \hat{G}(z|\theta) =k \frac{\prod (z-b_i) \prod(z^2 +b_jz + b_{j+1})}
  { \prod_{l=1}(z-a_l) \prod(z^2 +a_mz + a_{m+1})}z^{-p}. \label{eqn:mode_struc}
\end{equation}
Because the logarithm of the products is the sum of the logarithms, the Jacobian for the cost function is surprisingly easy to calculate. Details can be found in \cite{sidman_parametric_1991}.

The model structure \eqref{eqn:mode_struc} includes a fractional delay $z^{-p}$. This allows us to include the delay as a term in the decision variable and obviates the need to optimize over integers. This is further beneficial because we are not guaranteed that the latency from input to output is an exact integer and allowing a fractional delay helps the optimization to more accurately match the phase. In the final model, we round $p$ to the nearest integer. 

In this work, we do not model the modes above 1100~Hz. Thus, the optimization \eqref{eqn:logfit} is only done over frequency up to 1100~Hz. 

\begin{table}
  \centering
  \caption{Zero and pole locations of $G_{vib}$. }
  \begin{tabular}{cc}
    pole & zero\\
    0.848155$\pm$0.187900 & 0.964821 $\pm$ 0.250813\\ 
    0.962030$\pm$0.259730 & 0.992956 $\pm$ 0.093824\\ 
    0.972209$\pm$0.213448 & 0.975116 $\pm$ 0.209867\\ 
    0.978579$\pm$0.163808 & --\\ 
    0.993712$\pm$0.091498 & --\\ 
    0.917539 & 0.505483 \\ 
    (9) 0.0 & 0.822878 \\ 
  \end{tabular}
\end{table}


\subsection{Drift Modeling}\label{sec:drift_model}
Drift is modeled as the strictly proper transfer function
\begin{equation}
G_d = \theta_5\frac{(z-\theta_1)(z-\theta_2)}{(z-\theta_3)(z-\theta_4)}.\label{eqn:fit_drift_cost}
\end{equation}
We give the stage a step input with relatively small amplitude (to minimize the effects of hysteresis). The stage response is shown as the solid-blue curve in Fig.~\ref{fig:drift_fit} while the simulated response of the vibrational model is shown as the dotted-black curve. 

\begin{figure}
  \includesvg[width=1\linewidth]{figures/drift_fit.svg}
  \caption{Drift model}
  \label{fig:drift_fit}
\end{figure}
The goal then is to solve the non-linear least squares problem
\begin{equation}
  \min_{\theta}|| \mathcal{Z}^{-1}\left\{ \frac{a}{z-1}G_d(z|\theta)G_{vib}(z)\right\} - Y_{exp}||
\end{equation}
where $Y_{exp}$ is the experimental response of the stage, $\mathcal{Z}$ is the z-transform operator and $\theta$ is the vector of parameters. To the extent that $G_{vib}$ accurately models the vibtrational dynamics, the inclusion of $G_{vib}$ in \eqref{eqn:fit_drift_cost} effectively nullifies the vibrational aspects in the optimization.
The red curve in Fig.~\ref{fig:drift_fit} shows the simulated step response of $G_dG_{vib}$. 

It has been noted before in the literature that the rate of the creep affect itself is hysteretic \cite{Jung_open_loop_2000}, which I think in this context would imply that the real pole-zero pairs of $G_{drift}$ shift depending on the control history.

\subsection{Hysteresis Modeling}\label{sec:hyst_model}
There are many models for hysteresis. Here, we opt for simplicity (and by proxy, fast computation) and use the Prandtl-Ishlinksi Hysteresis model CITE. In typical raster scanning applications, hysteresis manifests as a bowing of the otherwise linear ramps. To motivate the need for hysteresis compensation in a step tracking application, consider Fig. REF, which shows an input signal of various steps applied open loop to the stage. The dotted-black curve is the input (scaled by the dc-gain $G_dG_{vib}$ while the solid black curve is the stage response. Observe that for the same steady state value of control, the steady state value of the stage is different. Effectively, the dc-gain of the system depends on the control history.


\begin{figure}
  \includesvg[width=1\linewidth]{figures/hyst_response_ol.svg}
  \caption{Evidence of hysteresis.}
  \label{fig:hyst_resp_dem}
\end{figure}


\subsection{Power Amplifier Characterization and Limitations}\label{sec:powcharct}
\begin{figure}[htbp]
\centering
\includesvg[width=1\linewidth]{figures/G_pow_and_current.svg}
\caption{\label{fig:orgc576458}
  Frequency responses for the power amplifier. The solid red curve is the transfer function from the low voltage command to the high voltage output $G_{V_X, u_X}$. The solid blue curve transfer function from high voltage output to current output. The solid black is the transfer function from low voltage control to power amplifier output current $G_{V_X, u_X}$, which is upper-bounded by the dotted-back curve representing $\tilde G_{I_X, u_x}$, a pure discrete derivative.}
\end{figure}

The high voltage output of the C300 is current limited to 100 mA. If the electrical impedance of the piezo is modeled as a pure capacitance (a common assumption  ~\cite{fleming_megahertz_2009, Bazghaleh_digital_2013}), this translates to a rate of change limitation on the voltage control signal through the relation $\dot v_{h} = \frac{1}{C} i_c(t)$. 

It is tempting to think that we ought to constrain the rate of change of the power amplifier voltage output $V_X$, since this is the voltage actually running through the piezo. There are two difficulties with this approach. First, as seen in Fig~\ref{fig:orgc576458}, \(G_{V_{x},u_{x}}\) rolls off at about 250~Hz, which means that such a constraint will lead to a constraint on the states of \(G_{V_{x},u_{x}}\). Second, the transfer function \(G_{I_{x}, V_{X}}\) has a break frequency at about 700~Hz, where it begins increasing at 40~dB per decade. Thus, a rate of change limitation based on a pure capacitance would \emph{under}-estimate the actual current draw.


In the interest of computational simplicity, we would like to approximate $G_{I_X,u_X}$ with a simple derivative, because this will lead to a box constraint on the rate of change of $u(k)$. Noting that, for frequency less than about 600 Hz, $G_{I_X,u_X}$ increases at 20dB per decade (i.e., looks like a pure derivative), we factor $G_{I_X,u_X}$ as
\begin{align}
  I_{pow} &= G_{I_X,u_X} u(z)\\
          & = (z-1) G_o u(z)\\
          & = G_o(z) \Delta u(z)
\end{align}
We now seek a bound such that
\begin{equation}
  |\Delta u(k)| < (\Delta u)_{max} \implies |I_{pow}| < I_{max}.
\end{equation}
Deriving such a bound is essentially the same as showing that $G_o$ is BIBO stable. We have in the time domain
\begin{align}
  |I_{pow}(k)| &= |\sum_{i=0}^{k} g_o(k-i)\Delta u(k)|\\
               &\leq ||\Delta u(k)||_{\infty} \sum_{i=0}^{k} |g_o(k-i)|  \\
               & \leq ||\Delta u(k)||_{\infty} \sum_{i=0}^{\infty} |g_o(i)|  \label{eqn:1norm}\\
               & = ||\Delta u(k)||_{\infty} ||g_o||_1  
\end{align}
Therefore, we require
\begin{equation}
|\Delta u(k)| \leq \frac{I_{max}}{||g_o||_1}
\end{equation}
To approximate $||g_o||_1$, we fit a model to $G_o(z)$, simulate the impulse response for a sufficiently long horizon, and compute the sum in \eqref{eqn:1norm}. This yields
\begin{align}
\Delta u(k))_{max} & \leq \frac{I_{max}}{||g_o||_1} = 0.1381. \label{eqn:du_limit}
% & = \frac{0.1}{ 0.7241 } 
\end{align}
where where $I_{\text{max}} = 0.1$~A and $||g_o||_1 \approx 0.724$.

Notably, this bound is conservative, since for higher frequencies a pure derivative overestimates the amount of actual current draw. Ideally, we would use our parametric model of \(G_{I_{X},u_{X}}\) and enforce a constraint on the output of that model. However, such an approach renders the constraint set for $u_X$ complex and is it not feasible to solve that problem using the fast gradient method. In our experimental implementations, we will enforce \eqref{eqn:du_limit}. However, in Section~\ref{sec:time_save_analysis} we will consider how much performance is given up through this method via a simulation study. 

Finally, the slew rate limit used in the MPC/linear feedback controller must be discounted from \eqref{eqn:du_limit} to account for the inverse inverse drift and hysteresis compensatators. This adjustment for the inverse drift operator is follows essentially the same argument as above. We have
\begin{equation}
\Delta y_d = G_d^{-1} \Delta u(z)
\end{equation}
and thus $|\Delta y_d| \leq |g_d^{-1}|_1 ||\Delta u||_{\infty}$. We find that $|g_d^{-1}|_1  = 1.39$ and thus, we need $|\Delta u| \leq 0.1381/1.39 = 0.0992$.

Computing a bound for the inverse hysteresis operator is more complicated. Recall the inverse operator is also an PI hysteresis operator and that it is composed of a linear combination of the fundamental elements REF. So start by rewriting the operation of a single element as

\begin{equation}
  z_i(k)  = 
  \begin{cases}
    u(k) + r_i, & u(k) \leq z_i(k-1) -r\\
    z_i(k-1), & z_i(k-1) - r < u(k) < z_i(k-1) + r_i\\
    u(k) - r_i, & u(k) \geq z_i(k-1) +r\\
  \end{cases}
\end{equation}
Then the increment in the output is given by
\begin{align}
  z_i(k+1)- z_i(k) &= \max\{u(k+1),\: \min\{z_i(k), u(k+1) - r_i\}\}\\
  &- \max\{u(k),\: \min\{z_i(k-1), u(k) - r_i\}\}.
\end{align}

TKTKTKTKT FINISH THIS

\section{Control Setup}
\subsection{The Incremental Form}\label{sec:incremental}
The constraint \eqref{eqn:du_limit} can be remodeled as a pure saturating constraint if we work with an incremental form of \(G_{X,u_x}\) which has as its input \({\Delta u(k)\coloneqq u(k)-u(k-1)}\), rather than \(u\). This is attractive because it not only allows us to directly penalize the rate of change in the optimal control problem but also renders the constraint a box constraint on $\Delta u$, enabling the use of the computationally attractive Fast Gradient Method. Putting a hard limit on $\Delta u$ in the linear feedback case is also somewhat simplified.

To this end, we augment \(G_{vib}\) with a state \(\x_{\text{u}}(k)\) such that
\begin{equation*}
  \x_u(k) = u(k-1).
\end{equation*}
It follows that
\begin{subequations}
\begin{align}
  \xd(k+1)
  &=
    \begin{bmatrix}
      A & B\\ 0 & I
    \end{bmatrix}
    \xd(k)
    +
    \begin{bmatrix}
      B\\I
    \end{bmatrix}
  \Delta u(k) \\
  \y(k) & = \begin{bmatrix}C & 0\end{bmatrix}\xd(k)\\
  \xd(k)& \coloneqq \begin{bmatrix}\x(k)\\\x_u(k) \end{bmatrix}\\
  \xd(0) & = \begin{bmatrix}\x(0)\\u(-1)\end{bmatrix}. \label{eqn:x0_aug}
\end{align}\label{eqn:ssdelta}%
\end{subequations}
We call this system \(\Gd = \{\Ad, \Bd, \Cd, 0\}\), which has \({\tilde{n}_s=21}\) states, 9 of which model delay.
To solve the setpoint tracking problem, we work in the error
coordinates of \(\Gd\).
For an arbitrary reference \(\y_r\), in steady state we have \({\du_{ss}=0}\) and \(\xdss =N_{\xd}\y_r\) where \({N_{\xd}\in\mathds{R}^{\tilde{n}_s\times n_u} }\) is found by solving
\begin{align}
  \begin{bmatrix}N_{\xd} \\ N_u\end{bmatrix} &=
\begin{bmatrix}I-\Ad & -\Bd\\\Cd & 0\end{bmatrix}^{-1}\begin{bmatrix}0\\ I\\\end{bmatrix}\label{eqn:nxnu},
\end{align}
which, due to the augmented pole at $z=1$, will give \(N_u\equiv 0\). 
The error state, \({\xde(k)=\xd(k) - \xdss}\) has dynamics
\begin{align}
  \xde(k+1) & = \Ad\xd(k) + \Bd\dd u(k) - \xdss \nonumber\\
            & = \Ad \xde(k)   + \Bd \dd u(k)\nonumber
\end{align}
because $\xdss$ is in the nullspace of $(I - \Ad)$.

% \section{The Optimal Control formulations}
We want to look at a few different ways of driving the stage from point to point as rapidly as possible, yet still be practical. Inspired by our past work, we focus on the experimental comparison of two methods:
\begin{itemize}
\item\emph{Saturated Linear state feedback (SLF)} The basic idea here to use standard linear state feedback which is saturated (on $\Delta u$) but to de-rate the design such that the closed-loop system is demonstrably stable up to a certain reference size. 
\item\emph{Constrained Model Predictive Control (MPC)} The goal with MPC is that, instead of fully de-rating the design to account for the slew rate constraint, to directly account for the constraint as part of the control law itself. Do to limitations on the implementation, MPC must still be de-rated to some extent to be implementable, since the length of control horizon needed to achieve stabilty over the full range of the stage becomes large for an aggressive design.
\end{itemize}

The state feedback controllers, MPC and linear state feedback, can both be described by the optimization
\begin{align}
V(v) &= \min_{v} z_{N}Q_{p}z_{N} + \sum_{i=0}^{N-1}z_{i}^{T}Qz_{i} + z^T_iSu_i + v^{T}_{i}Rv_{i} \label{eqn:optcost}\\
 \text{s.t.} & \nonumber\\
z_{i+1} &= Az_{i} + Bv_{i}\\
z_{0} &= x_{k}\\
|v_i | & \leq v_{max}.\label{eqn:cntrl_constraint}
\end{align}
where $Q$ and $R$ are symmetric matrices and the matrices $Q,R,S$ satisfy
\begin{equation}
  \begin{bmatrix}
    Q & S\\S^T &R
  \end{bmatrix} > 0
\end{equation}
Here, $Q_p$ is the solution of the discrete algebraic Ricattii equation. The solution of \eqref{eqn:optcost} results in a sequence of length $N$ of optimal controls, $\{v_i\}_{i=0}^{N-1}$. In the case of MPC, one sets $\Delta u(k) = v_0$ and discards the remaining elements. The process is repeated at the next time step. Note that in the case of MPC, the saturator in Fig. \ref{fig:ss_bd} is superfluous because the optimal control is guarnteed to satisfy the constraints.

In the case of linear feedback, $N=1$ and we eliminate the constraint on the control \eqref{eqn:cntrl_constraint} and thus the optimal control is simply $Kx_k$, where $K$ is LQR feedback gain associated with $Q, R,S$. Here, 

\subsection{Control Weight Selection}
To design the weighting matrices $Q$, $R$ and $S$, we design fictitious output matrices $\hat C$ and $\hat D$ such that zeros of $\{A, B, \hat C, \hat D\}$ are at the desired pole locations. Then, the fictitious output is $\hat y = \hat C x + \hat D u$. If we write the cost function of REF as
\begin{align}
  J &= \sum_{i=0}^{N-1} \hat{y_i}^T\hat{y_i} + u_i^T\gamma u_i\\
    &= \sum_{i=0}^{N-1} \hat{x_i}^TQ\hat{x_i} + 2x_i^TSu_i + u_i^T(R_o+\gamma)u_i
\end{align}
where $Q = \hat{C}^T\hat{C}$, $S =\hat{C}^T \hat{D}$, and $R_o = \hat{D}^T\hat{D}$. As $\gamma$ becomes small, the closed loop poles of the unconstrained LQR will move to the zeros of $\{A, B, \hat C, \hat D\}$. This is illustrated in Fig.~\ref{fig:lqr_locus}. One can usually take $\gamma$ to be small enough that the difference between the desired pole location and its actual location is negligible. The direct feedthrough term $\hat D$ results in the cross-weighting term $S$ and is necessary if we wish to endow the fictitious system with $n_s$ zeros in order to place all $n_s$ poles.

Through elementary block row and column operations, it is straightforward to show that the zeros of
$\{A, B, \hat C, \hat D\}$ are the same as the generalized eigenvalues of
\begin{equation}
  \begin{bmatrix}
    (A - B\hat{D}^{-1}\hat{C}) & 0\\
    0 & \hat{D}
  \end{bmatrix}
\end{equation}
Thus, we choose $\hat D = 1$ and find $\hat C$ via standard pole placement techniques.

\begin{figure}
  \includesvg[width=.5\textwidth]{figures/lqr_locus.svg}
  \caption{Root locus for $R_o + \gamma$. Note that for clarity, the plant zeros are not shown. The black 'x's indicate the poles of the open-loop plant. The blue circles indicate the fictitious zeros, which are at the location of the desired poles. }
  \label{fig:lqr_locus}
\end{figure}

Certainly, in the case of the LQR state feedback controller, one could simply use a pole-placement design to start with. However, the method here has two advantages: (1) it permits a straightforward comparison to the MPC design (which requires weighting matrices, not simple pole locations) and (2) the design becomes paramtrized by the single parameter $\gamma$ which makes de-rating the design easy. In contrast, with standard pole-placement it is not clear how to ``back-off'' the design if the slew rate constraint is violated to the extent that instability results. 


\subsection{Observer Design}\label{sec:dist_est}
To achieve zero-offset tracking (to constant disturbances), we employ the disturbance estimator outlined in \cite{maeder_offset-free_2007}. The disturbance dynamics are modeled as a pure integrator and the system matrices are augmented as
\begin{equation}
  A_m = \begin{bmatrix}
    A & B_d \\ 0 & I
  \end{bmatrix}\quad
  B_m =
  \begin{bmatrix}
    B \\ 0
  \end{bmatrix}
  C_m = 
    \begin{bmatrix}
    C \\ C_d
  \end{bmatrix}
\end{equation}
The observer gain is given by $\begin{bmatrix}L & L_d\end{bmatrix}$. It is shown in \cite{maeder_offset-free_2007} that the gains $L$ and $L_d$ may be designed separately and then transformed such that the closed loop poles $A_m - L_mC_m$ are the same as those of $A-LC\cup A_d-L_dC_d$.

To design $L$, we use the method outlined in \cite{doyle_robustness_1979}, which sets $L$ is the steady state solution of the discrete LQR problem applied to the dual of $G$, where $Q = Q_w + BB^T$ and $R$ is a turning parameter. We design $L_m$ such that the disturbance pole is placed at $z=0.8$.

To achieve zero offset tracking, disturbance estimators re-compute the steady-state target at each time-step. For an output disturbance ($B_d=0$ model with $C_d=I$, this simply means that the reference is adjusted by subtracting $\hat d_k$. In other words, at each time step, we need to compute
\begin{equation}
  x_{e} = \hat{x}(k) - N_x(r_k - \hat{d}_k).
\end{equation}
This is slightly simpler than the case for an input disturbance model ($C_d=0$), which involves an additional vector-scalar multiplication and an additional vector-vector addition (see (21) in \cite{maeder_offset-free_2007}). It is shown in \cite{maeder_offset-free_2007} that output disturbance and input disturbance models are equivalent provided 1 is not an eigenvalue of $A$, which is the case here, because we do not estimate the state $x_u$. Thus, due to the computational savings, we use the output disturbance formulation. 
\begin{figure}
  \includesvg[width=1\linewidth]{figures/obs_cl.svg}
  \caption{(black) Open loop plant pole (x) and zero(circle) locations. (red x) Closed loop observer pole locations.}
  \label{fig:obs_cl}
\end{figure}

\section{Quantifying the time savings (if any) of MPC and de-rated linear feedback}
\label{sec:org7edf36d}
The goal of this section is to quantify the time savings (if any) of using MPC vs SLF. To make things definite, throughout this section, we consider $Q, R_o, S$ to be fixed and consider $\gamma$ as the tuning parameter.

We make the following observations
\begin{itemize}
\item The stabilizable range of setpoints for both MPC and SLF can be increased by relaxing the performance TKT (i.e., by increasing $\gamma$.
\item The stabilizable range of setpoints for MPC can be increased by increasing the control horizon $N$.
\end{itemize}

In the ideal sense, we would make the MPC control horizon long enough (say, $N=200$) that for every setpoint the stage is capable of visiting we could guarantee that $u_N$ is inside the feasible set. This comes with good theoretical guarantees of stability CITE. Unfortunately, solving such a scheme is too slow and resource intensive to solve at the required sample rate with our hardware. Nonetheless, stable trajectories still result for much shorter control horizons. 

Conversely, for both SLF and MPC, one could increase $\gamma$ enough that the constraint is never violated past $N$. However, this results in an excessively docile control law.

As we will see (and as one might expect), SLF requires more de-rating than MPC and with MPC, longer control horizons require less de-rating than shorter ones. Thus we seek to investigate two related things. The first is to investigate the mapping of $\gamma$ to range of stabilizable setpoints for SLF and different control horizons of MPC. The second is to consider, for a given range setpoints (and thus required $\gamma$s), how the settling times compare for SLF and MPC with different control horizons.

To be able to make this comparison, we need a way to decide when a trajectory is unstable, or more generally, unacceptable. There are a number of ways one might approach this. The most naive way to this would be to look at the settling time over a sufficiently long simulation: if the trajectory never settles into the settle boundary, we decide the system was unstable for that setpoint size. The trouble with this is that there is a middle ground, where the trajectory will actually settle, but spends a lot of time oscillating (in some non-linear fashion) before it gets there. This is illustrated in Fig. REF. 

 

% Additionally, in simulation we consider two additional techniques:
% \begin{itemize}
% \item\emph{Discrete time, minimum time optimal control (DTMT)}
% \item\emph{Constrained, Finite horizon LQR (CLQR)} The optimization is the same as the MPC optimization, just over a much longer horizon. Similarly to the DTMT idea, this would be implemented in a feedforward/tracking configuration. However, the broader goal of considering this scheme is that it shows how much optimality we have sacrificed with a relatively short horizon in MPC.
% \end{itemize}
% These schemes are used as benchmarks against which to judge the performance of SLF and MPC. Certainly, one could apply the control trajectories either CLQR or DTMT in a feedforward manner and track the optimal state sequences, as we have considered before in past work \cite{braker_fast_2017}. However, such a technique has several disadvantages: (1) the entire sequence of setpoints must be known before hand, eliminating the possibility of adjusting them in real time. (2) Computing each optimal trajectory requires defining an initial condition, which becomes problematic when, for example, performing short $\mu$-path scans and (3) transferring the data defining each trajectory in real-time increases the burden on the I/O bus.

\subsection{Rejection Metrics}\label{sec:metrics}
There are several ways we might think about doing this. 
For a given $\gamma$, we need a way to decide if the trajectory which results from either SLF or MPC is acceptable for a given setpoint size.

Ultimately, we have decided to do is to compare the trajectory resulting from MPC or SLF with the one from CLQR: if the settling time is greater than 20\% of the CLQR trajectory, we decide that the current setpoint was too large.


\begin{enumerate}
\item We start with a set of nominal weights \((Q,R, S)\) pair which, when used in the discrete (unconstrained) LQR will put the closed loop eigenvalues of $(sI - A + BK_{lqr})$ at the nominal desired locations. 
\item We determine how large we must make \(\gamma R\) to visit a certain size of setpoint. In general, the larger \(\gamma\) is, the larger the maximum setpoint we can visit is. This is what I mean by "de-rating" the feedback gain and is illustrated below in Figure. REF
\end{enumerate}


Note that each point in that plot was generated by simulated the linear feedback law with saturation starting from a very small setpoint and incrementing larger (by 0.1 I think) until the output becomes unstable. 

So the questions I want to answer are:
\begin{enumerate}
\item Given a maximum setpoint \(ref_{max}\), how much time do we \textbf{loose} by making \(\gamma\) large? Said another way, what is the time penalalty for needing to visit larger setpoints (since this implies that \(K\) must be "smaller")?
\item How does this time compare to what we can acheive with MPC?
\end{enumerate}


The way I think it makes sense to think about these questions is to compare the settling times to the optimal, constrained finite horizon (CLQR) open-loop settling time. At least for a given quadratic cost, that is typically the best settling time we can hope to achieve, and we won't run into the problems with MPC where need a larger control horizon to get stability. 

\begin{figure}
  \includesvg[width=1\linewidth]{figures/maxref_vs_gamma_dumaxp6.svg}
  \caption{The increase in maximum achievable setpoint as the control weight $\gamma$ increases for both saturated linear feedback different MPC control horizons.}
  \label{fig:maxsp_by_gam}
% \hfill
% \subfigure[Comparison of settling time the minimum-time solution and the CLQR solution.
% \label{fig:ts_clqr_to}]
% {\includesvg[width=3in]{figures/clqrTimeOpt_sp_vs_ts_CCTA_dumaxp6.svg}}
% \caption{both}
\end{figure}

The result of running a parameter sweep this way is shown in Fig.~\ref{fig:maxsp_by_gam}. What the figure shows is, for SLF, and several values of the MPC control horizon, the maximum setpoint that can be tracked for a given control weight $\gamma$. In the figure $\gamma$ ranges from 10 to 5000 in XX increments. For each value of $\gamma$, we performed successively larger step-input simulations between and 12 v. The $y$-axis value indicates that that value of reference resulted in a deteriorated or unstable trajectory, and is thus the largest track-able setpoint for the given gamma.

\section{Time Savings Analysis}\label{sec:time_save_analysis}
The goal of doing this is to essentially look at several cross sections Fig.~\ref{fig:maxsp_by_gam} and look at how the settling time changes between SLF and different MPC control horizons \emph{for a given desired maximum setpoint.}

Here, we choose several desired $r_{max}$ values and consider how much time is saved over the time optimal solution for the different strategies. We choose $r_{max}$ as 2.5, 5.0 and 10. These values are illustrated in Fig.~\ref{fig:maxsp_by_gam} by the dotted black lines. Where those dotted black lines cross the various curves indicates the $\gamma$ that is required.

\begin{figure*}
  \begin{minipage}{0.45\textwidth}
    \includesvg[width=1.2\textwidth]{figures/perc_increase_Ipow_lowgain_rmax2p5.svg}
  \end{minipage}
  % \begin{minipage}{0.32\textwidth}
  %   \includesvg[width=1.2\textwidth]{figures/perc_increase_Ipow_lowgain_rmax5.svg}
  % \end{minipage}
  \begin{minipage}{0.45\textwidth}
    \includesvg[width=1.2\textwidth]{figures/perc_increase_Ipow_lowgain_rmax10.svg}
  \end{minipage}
  \caption{The percentage increase over the time-optimal solution for SLF (orange), the CLQR solution (blue) and the MPC strategy for control horizons ranging from $N=4$ to $N=20$. }
  \label{fig:perc_inc}
\end{figure*}
Our takeaway from this study is that there appears to very little benefit in terms of time saved for implemented the CLQR or MPC control strategies. The largest difference in the percent increase is only in the range of 10\%, and for the smaller $r_{max}=2.5$, this difference is even smaller. 

\section{Experimental Comparison}\label{sec:exp_comp}
Our simulation results indicate that we may as well use SLF or track the time-optimal trajectory, if we can get away with it. But what about the real world?



\begin{table*}
  % \begin{tabular}{cccccc}
    \input{manystepsdata.tex}
  % \end{tabular}
\end{table*}


\begin{table*}
  % \begin{tabular}{cccccc}
    \input{manystepsdata_rand.tex}
  % \end{tabular}
\end{table*}


Table ~\ref{tab:resource} compares the FPGA resource utilization of the different control schemes.
\begin{table*}
  \caption{Resource utilization of the different control schemes. Each column shows the total amount of the resource used as well as the percent of the total available.}
  \begin{tabular}{cccccc}
    implementation & DSP48 & BRAM & LUT   & Reg & Total Slices \\
    \toprule
    MPC            & 75 (41.7\%)   & 24 (9.0\%)  & 46860 (50.9\%) & 56489 (30.6\% &   18133  (78.7\%)\\
    Lin FXP        & 31 (17.2\%)   & 13 (4.9\%)  & 29094  (31.6\%)& 30154  (16.4\%)& 10871  (47.2\%)\\
    PI             & 5  (2.8\%)    & 4  (1.5\%)  & 9722   (10.5\%)& 7546   (4.1\%) & 3485  (15.1\%)\\
    Lin FP         & 34 (18.9\%)   & 10 (3.7\%)  & 34864  (37.8\%)& 26412  (14.3\%)& 11431 (49.6\%)\\
  \end{tabular}
  \label{tab:resource}
\end{table*}

\newpage

% \FloatBarrier
\section{Imaging Results}\label{sec:imaging_results}
% In these section, we demonstrate using the time-optimal tracking method to actually acquire images via $\mu$-path based CS.
% \begin{figure}
%   \centering
% \includegraphics[scale=1]{figures/5micron_rasterscans_v2.eps}
% \includegraphics[scale=1]{figures/5micron_csscans_v2.pdf}
% \caption{RESULT!}
% \label{fig:imaging}
% \end{figure}

\bibliographystyle{IEEEtran}
\bibliography{/home/arnold/bib_pdf/main_bibliography.bib}

\renewcommand{\theequation}{A-\arabic{equation}}
% redefine the command that creates the equation no.
\setcounter{equation}{0}  % reset counter 
\setcounter{section}{0}
\section*{Appendix: The minimum-time control problem in discrete-time}\label{sec:mintime}
In discrete-time, the minimum-time control problem can be stated as \cite{chen_minimumtime_cca}
\begin{align}
\min_{u(0), u(1),\dots,u(N-1)} & N\\
\text{s.t.}\quad
x_{k+1} & = Ax_{k} + Bu_{k}\\
x_{N} & = x_{f}\\
u_{k}&\in \mathds{K},~k=0,\dots,N-1\\
x_{k} &\neq x_{f},~k=0,\dots,N-1.
\end{align}
where $x_f$ is the desired final target state. 
For a given control sequence $\{u_0,\dots, u_{N-1}\}$, and an initial condition $x_0$, the state $x_N$ at the final time is given by
\begin{equation}
x_N = A^Nx_0 + \sum_{i=0}^{N-1}A^{N-1-i}Bu_i.
\end{equation}
Practically, the problem can be solved using a bisection method which searches for the smallest feasible \(N\). In these cases, the bisection search solves sub-problem given by
\begin{align}
\min_{U}& || x_{f} - x_{N}||\\
\text{s.t.} &\\
x_{k+1} & = Ax_{k} + Bu_{k}\\
u_{k}&\in \mathds{K},~k=0,\dots,N-1\\
\end{align}
For a given \(N\), if \(||x_{f} - x_{N}|| < \text{TOL}\), the sub-problem returns successful, otherwise it fails. The goal then is to find a the smallest \(N\) that succeeds. 

It is also worth pointing out that the solution to this problem, in contrast to continuous time systems is not in general bang-bang CITE.
\end{document}